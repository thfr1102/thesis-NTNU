\chapter{Results}
\label{chap:results}
This chapter will introduce the results of the data analysis. I will touch on some topics for discussion for some of the results, however most of the discussion will take place in chapter \ref{chap:discussion}. This chapter will start by describing the demographics and background of the main sample. Further, I will present the main findings that assess the security awareness of the respondents. Next I present my findings from the bivariate analysis, and lastly present and compare the results from my control group sample. 

\section{Demographics}

The following section will describe the demographics of my samples, as well as compare relevant groups to each other and the Norwegian national population. Out of the total of 222 people who answered, one person did not want to specify any demographic data. 

\subsection{Age}
The age distribution of my sample. None of the respondents reported being under 20 years old. 33 (14.9\%) of the respondents answered being between 20 and 29 years old and a whole 102 (45.9\%) responded that they are between 30 and 39 years old, which is the largest age group of our sample. Further, 56 (25.2\%) people answered being between 40 and 49 years old, which is the second largest age group. Lastly, 24 (10.8\%) of my sample is between 50 and 59 years old and 6 (2.7\%) people being 60 or older. 

\begin{figure}[H]
    \centering
    \includegraphics[scale=0.45]{figures/diagrams/age_ssb.pdf}
    \caption{Age distribution}
    \label{fig:age}
\end{figure}

As we can see from figure \ref{fig:age} above, we have a fairly middle-aged sample compared to the national population, which has a sizeable difference in younger and older people compared to my sample. 

Since the number of respondents being 60 or older in my sample only comprise 6 respondents, I will merge that category with 50-59 years going forward, resulting in a single category of 50 or older. This will result in the category comprising 30 respondents, which should be the minimum for further analysis between the groups. 

\subsection{Gender}
The gender distribution of my main sample is all men. There were also one respondent who did not want to specify, however there are no women in my sample. This makes it impossible to use in further analysis other than for sample description. This is obviously a huge deviation from the national average of approximately 50\% of each gender. 


\subsection{Highest completed education level}
None of the respondents answered that they had only primary school education or no education, and only one person did not want to specify education level. 48 people (21.6\%) answered that they has completed high school, while 55 of the respondents (24.8\%) has completed vocational college. 76 people (34.2\%) has completed at least four year university or college, and 42 (18.9\%) has completed university or college for longer than four years. 

\begin{figure}[H]
    \centering
    \includegraphics[scale=0.45]{figures/diagrams/education_ssb.pdf}
    \caption{Highest completed education level}
    \label{fig:education}
\end{figure}

From figure \ref{fig:education} above we can see that my sample contain a higher share of people with higher education. The most interesting fact is that the data shows a huge difference when it comes to people reporting to having a vocational education. 

\subsection{County}
There were two people who did not want to specify which county they lived in. The sample distribution in figure \ref{fig:county} below shows that it is very close to the national distribution. The only significant outlier is that my sample has a bit more people from Rogaland (14\%), compared to the national level (9\%).

\begin{figure}[H]
    \centering
    \includegraphics[scale=0.45]{figures/diagrams/county_ssb.pdf}
    \caption{County population distribution}
    \label{fig:county}
\end{figure}

It should be mentioned that most of the categories from my sample has less than 30 answers, which might impact the comparability of the numbers. For example, Troms og Finnmark only has 6 respondents in my sample, and Nordland has 9 respondents. At the other end of the spectrum lies Viken with 50 respondents and Rogaland with 31 respondents. 

\section{Background}

\subsection{How smart are the homes?}
In order to investigate how invested people are in the smart home ecosystem, I chose to include a question that aimed to assess which smart device types the respondents owns. In table \ref{tab:how_smart_N} below is an overview of the number of respondents that answered this question. 

\begin{table}[H]
\centering
\begin{tabular}{|l|c|c|c|c|c|c|}
\hline
\multicolumn{7}{|c|}{{\color[HTML]{010205} \textbf{Case summary}}}                                                                                                                                                                                                                                                                                                              \\ \hline
{\color[HTML]{264A60} }                                      & \multicolumn{6}{c|}{{\color[HTML]{264A60} Cases}}                                                                                                                                                                                                                                                                \\ \cline{2-7} 
{\color[HTML]{264A60} }                                      & \multicolumn{2}{c|}{{\color[HTML]{264A60} Valid}}                                                    & \multicolumn{2}{c|}{{\color[HTML]{264A60} Missing}}                                               & \multicolumn{2}{c|}{{\color[HTML]{264A60} Total}}                                                     \\ \cline{2-7} 
\multirow{-3}{*}{{\color[HTML]{264A60} }}                    & {\color[HTML]{264A60} N}                        & {\color[HTML]{264A60} Percent}                     & {\color[HTML]{264A60} N}                      & {\color[HTML]{264A60} Percent}                    & {\color[HTML]{264A60} N}                        & {\color[HTML]{264A60} Percent}                      \\ \hline
\cellcolor[HTML]{E0E0E0}{\color[HTML]{000000} Smart devices} & \multicolumn{1}{r|}{{\color[HTML]{010205} 219}} & \multicolumn{1}{r|}{{\color[HTML]{010205} 98.6\%}} & \multicolumn{1}{r|}{{\color[HTML]{010205} 3}} & \multicolumn{1}{r|}{{\color[HTML]{010205} 1.4\%}} & \multicolumn{1}{r|}{{\color[HTML]{010205} 222}} & \multicolumn{1}{r|}{{\color[HTML]{010205} 100.0\%}} \\ \hline
\end{tabular}
\caption{Number of people who specified what smart device types they own}
\label{tab:how_smart_N}
\end{table}

Out of the three that did not answer, two of them specified in free text that they used a KNX system with control of heating, lighting, and ventilation, as well as motion sensors among other things. Multiple people who answered, also specified in free text that they used a KNX system, and many others specified that they had a smart home with basically ``everything''. One respondent answered that they had no smart devices. The table below shows the frequency of which types of smart devices the respondents owned. 
\begin{table}[H]
\centering
\begin{tabular}{|l|l|r|r|r|}
\hline
\multicolumn{5}{|c|}{\textbf{Smart device types frequencies}}                                                                                                                                                                                                                                                                           \\ \hline
\multicolumn{2}{|l|}{}                                                                                                                           & \multicolumn{2}{c|}{{\color[HTML]{264A60} Responses}}                                               & \multicolumn{1}{c|}{{\color[HTML]{264A60} }}                                   \\ \cline{3-4}
\multicolumn{2}{|l|}{\multirow{-2}{*}{}}                                                                                                         & \multicolumn{1}{c|}{{\color[HTML]{264A60} N}} & \multicolumn{1}{c|}{{\color[HTML]{264A60} Percent}} & \multicolumn{1}{c|}{\multirow{-2}{*}{{\color[HTML]{264A60} Percent of Cases}}} \\ \hline
\cellcolor[HTML]{E0E0E0}{\color[HTML]{000000} }                                 & \cellcolor[HTML]{E0E0E0}{\color[HTML]{264A60} Voice assistant} & {\color[HTML]{010205} 155}                    & {\color[HTML]{010205} 6.7\%}                        & {\color[HTML]{010205} 70.8\%}                                                  \\ \cline{2-5} 
\cellcolor[HTML]{E0E0E0}{\color[HTML]{000000} }                                 & \cellcolor[HTML]{E0E0E0}{\color[HTML]{264A60} Speaker}         & {\color[HTML]{010205} 158}                    & {\color[HTML]{010205} 6.8\%}                        & {\color[HTML]{010205} 72.1\%}                                                  \\ \cline{2-5} 
\cellcolor[HTML]{E0E0E0}{\color[HTML]{000000} }                                 & \cellcolor[HTML]{E0E0E0}{\color[HTML]{264A60} Robot vaccum}    & {\color[HTML]{010205} 123}                    & {\color[HTML]{010205} 5.3\%}                        & {\color[HTML]{010205} 56.2\%}                                                  \\ \cline{2-5} 
\cellcolor[HTML]{E0E0E0}{\color[HTML]{000000} }                                 & \cellcolor[HTML]{E0E0E0}{\color[HTML]{264A60} Smart hub}       & {\color[HTML]{010205} 169}                    & {\color[HTML]{010205} 7.3\%}                        & {\color[HTML]{010205} 77.2\%}                                                  \\ \cline{2-5} 
\cellcolor[HTML]{E0E0E0}{\color[HTML]{000000} }                                 & \cellcolor[HTML]{E0E0E0}{\color[HTML]{264A60} Smart TV}        & {\color[HTML]{010205} 185}                    & {\color[HTML]{010205} 8.0\%}                        & {\color[HTML]{010205} 84.5\%}                                                  \\ \cline{2-5} 
\cellcolor[HTML]{E0E0E0}{\color[HTML]{000000} }                                 & \cellcolor[HTML]{E0E0E0}{\color[HTML]{264A60} Smart screen}    & {\color[HTML]{010205} 69}                     & {\color[HTML]{010205} 3.0\%}                        & {\color[HTML]{010205} 31.5\%}                                                  \\ \cline{2-5} 
\cellcolor[HTML]{E0E0E0}{\color[HTML]{000000} }                                 & \cellcolor[HTML]{E0E0E0}{\color[HTML]{264A60} Router}          & {\color[HTML]{010205} 143}                    & {\color[HTML]{010205} 6.2\%}                        & {\color[HTML]{010205} 65.3\%}                                                  \\ \cline{2-5} 
\cellcolor[HTML]{E0E0E0}{\color[HTML]{000000} }                                 & \cellcolor[HTML]{E0E0E0}{\color[HTML]{264A60} Door lock}       & {\color[HTML]{010205} 149}                    & {\color[HTML]{010205} 6.5\%}                        & {\color[HTML]{010205} 68.0\%}                                                  \\ \cline{2-5} 
\cellcolor[HTML]{E0E0E0}{\color[HTML]{000000} }                                 & \cellcolor[HTML]{E0E0E0}{\color[HTML]{264A60} Light bulbs}     & {\color[HTML]{010205} 163}                    & {\color[HTML]{010205} 7.1\%}                        & {\color[HTML]{010205} 74.4\%}                                                  \\ \cline{2-5} 
\cellcolor[HTML]{E0E0E0}{\color[HTML]{000000} }                                 & \cellcolor[HTML]{E0E0E0}{\color[HTML]{264A60} Smart dimmer}    & {\color[HTML]{010205} 181}                    & {\color[HTML]{010205} 7.8\%}                        & {\color[HTML]{010205} 82.6\%}                                                  \\ \cline{2-5} 
\cellcolor[HTML]{E0E0E0}{\color[HTML]{000000} }                                 & \cellcolor[HTML]{E0E0E0}{\color[HTML]{264A60} Smart switch}    & {\color[HTML]{010205} 177}                    & {\color[HTML]{010205} 7.7\%}                        & {\color[HTML]{010205} 80.8\%}                                                  \\ \cline{2-5} 
\cellcolor[HTML]{E0E0E0}{\color[HTML]{000000} }                                 & \cellcolor[HTML]{E0E0E0}{\color[HTML]{264A60} Kitchenware}     & {\color[HTML]{010205} 41}                     & {\color[HTML]{010205} 1.8\%}                        & {\color[HTML]{010205} 18.7\%}                                                  \\ \cline{2-5} 
\cellcolor[HTML]{E0E0E0}{\color[HTML]{000000} }                                 & \cellcolor[HTML]{E0E0E0}{\color[HTML]{264A60} Surveillance}    & {\color[HTML]{010205} 138}                    & {\color[HTML]{010205} 6.0\%}                        & {\color[HTML]{010205} 63.0\%}                                                  \\ \cline{2-5} 
\cellcolor[HTML]{E0E0E0}{\color[HTML]{000000} }                                 & \cellcolor[HTML]{E0E0E0}{\color[HTML]{264A60} Alarms}          & {\color[HTML]{010205} 111}                    & {\color[HTML]{010205} 4.8\%}                        & {\color[HTML]{010205} 50.7\%}                                                  \\ \cline{2-5} 
\cellcolor[HTML]{E0E0E0}{\color[HTML]{000000} }                                 & \cellcolor[HTML]{E0E0E0}{\color[HTML]{264A60} Motion sensors} & {\color[HTML]{010205} 177}                    & {\color[HTML]{010205} 7.7\%}                        & {\color[HTML]{010205} 80.8\%}                                                  \\ \cline{2-5} 
\multirow{-16}{*}{\cellcolor[HTML]{E0E0E0}{\color[HTML]{000000} Smart devices}} & \cellcolor[HTML]{E0E0E0}{\color[HTML]{264A60} Thermostat}      & {\color[HTML]{010205} 171}                    & {\color[HTML]{010205} 7.4\%}                        & {\color[HTML]{010205} 78.1\%}                                                  \\ \hline
\multicolumn{2}{|l|}{\cellcolor[HTML]{E0E0E0}{\color[HTML]{264A60} Total}}                                                                       & {\color[HTML]{010205} 2310}                   & {\color[HTML]{010205} 100.0\%}                      & {\color[HTML]{010205} 1054.8\%}                                                \\ \hline
\end{tabular}
\caption{What types of smart devices the respondents own}
\label{tab:how_smart}
\end{table}

Here, N shows how many responses of each category there was. People were allowed to make multiple choices, so the total amount of answers amounts to 2310. Considering that there were 219 people who answered this, we can see that on average every respondent chose a little over ten device types. The percent of cases shows us the percent of the respondents who chose each category. We observe that it was very common for the respondents to have a Smart TV (84\%), smart dimmers (82.6\%) and switches (80.8\%), as well as motion sensors (80.8\%). At was however uncommon for people to have smart kitchenware, with only 18.7\% responses. Only two categories was not chosen by at least 50\% of the respondents. 

There were also several people who specified further devices in the free text section. 


\subsection{Household smart home administrators}
I asked a question regarding whether or not the respondents were the administrators of their smart home. My hypothesis in advance was that the large majority were the smart home administrators of their household solely due to the fact that I collected my sample from a social media group of smart home enthusiasts. As we can see from figure \ref{fig:administrator} below, this turned out to be correct. 

\begin{figure}[H]
    \centering
    \includegraphics[scale=0.55]{figures/diagrams/administrator.pdf}
    \caption{Household administrators of their smart home}
    \label{fig:administrator}
\end{figure}

Out of the total of 222 people who answered, 219 (98.6\%) of the respondents said that they were the smart home administrator of their household. Only 2 people said no, and the last one said that they did not know. This shows us that the people in this study are active users, and not passive ones. 

\subsection{Professional / hobby based background}
I asked a question to figure out if the respondents had a background in technology, either a professional one or as a hobby. My hypothesis was that most people would have a background in technology, since that is how you get exposed to and interested in devices like those in a smart home. From figure \ref{fig:background} below, we can see that most people does have a background in technology. 

\begin{figure}[H]
    \centering
    \includegraphics[scale=0.55]{figures/diagrams/background.pdf}
    \caption{The respondents background in IT or technology}
    \label{fig:background}
\end{figure}
Out of the total of 222 people who answered this question, 177 respondents (79.7\%) said yes and 41 people (18.5\%) answered no. The last four people said that they did not know. This shows us that the sample consist of many people from a technological background. 

\subsection{Knowledge of subjects}
In addition to the previous questions, I also wanted to assess the respondents knowledge on certain subjects relating to smart home security. This is of course only self-reported knowledge. My hypothesis was that they know a lot about technology and smart homes, but not that much about data security. We can see that the hypothesis was mostly correct based on figure \ref{fig:knowledge} below. 

\begin{figure}[H]
    \centering
    \includegraphics[scale=0.3]{figures/diagrams/knowledge.pdf}
    \caption{The respondents knowledge of three different subjects}
    \label{fig:knowledge}
\end{figure}

159 (71.6\%) of the respondents answered that they know technology well, while another 51 (23\%) claims to know it. This amounts to 94.6\% of the respondents alone. Regarding data security, 90 people (40.5\%) said they know it well, while 88 people (39.6\%) responded that it was known to them. This is considerable less than with technology, but still fairly good as it amounts to around 80.2\%. Lastly, 114 respondents (51.4\%) answered that smart homes was well known to them, while 79 people (35.6\%) responded that it was known. This is more than data security, however still significantly less than technology. While still being partly correct in my hypothesis, it was surprising that the knowledge of smart homes was not closer to technology for a sample specifically interested in smart homes. This could have many explanations, such that there are people that joined the Facebook group in order to learn, so they might not be experts yet. Another explanation could be that they underestimate their own expertise, and overestimate what they do not know. This has been proven to be the case in many studies previously \cite{Ehrlinger2008} \cite{MCCORMICK1986205}. 

\section{Security awareness of the respondents}

\subsection{Use of Smart Home Devices}

\begin{figure}[H]
    \centering
    \includegraphics[scale=0.55]{figures/diagrams/update.pdf}
    \caption{The respondents routines towards updates their devices}
    \label{fig:update}
\end{figure}



\begin{figure}[H]
    \centering
    \includegraphics[scale=0.55]{figures/diagrams/turn_off_features.pdf}
    \caption{The respondents routines towards turning off features and services they do not use}
    \label{fig:turn_off_features}
\end{figure}

\begin{figure}[H]
    \centering
    \includegraphics[scale=0.55]{figures/diagrams/separat_segment.pdf}
    \caption{The respondents routines towards connecting smart devices to a separate home network segment}
    \label{fig:separat_segment}
\end{figure}

\begin{figure}[H]
    \centering
    \includegraphics[scale=0.55]{figures/diagrams/settings.pdf}
    \caption{The respondents routines towards changing their security and privacy settings}
    \label{fig:settings}
\end{figure}

\begin{figure}[H]
    \centering
    \includegraphics[scale=0.55]{figures/diagrams/connect_internet.pdf}
    \caption{The respondents preference for cable or wireless when connecting their smart devices to the Internet}
    \label{fig:connect_internet}
\end{figure}

\subsection{Credential management}
\begin{figure}[H]
    \centering
    \includegraphics[scale=0.55]{figures/diagrams/standard_password.pdf}
    \caption{The respondents routines towards changing standard passwords}
    \label{fig:standard_password}
\end{figure}

\begin{figure}[H]
    \centering
    \includegraphics[scale=0.55]{figures/diagrams/standard_password.pdf}
    \caption{The respondents routines towards using password managers}
    \label{fig:standard_password}
\end{figure}

\begin{figure}[H]
    \centering
    \includegraphics[scale=0.55]{figures/diagrams/password_reuse.pdf}
    \caption{The respondents routines towards using password on multiple devices/services}
    \label{fig:password_reuse}
\end{figure}

\subsection{Knowledge of smart home security aspects}

\begin{figure}[H]
    \centering
    \includegraphics[scale=0.45]{figures/diagrams/knowledge_security.pdf}
    \caption{Knowledge of different security aspects relating to smart homes}
    \label{fig:knowledge_security}
\end{figure}

\subsection{Risk perceptions of the respondents}

\begin{figure}[H]
    \centering
    \includegraphics[scale=0.4]{figures/diagrams/risk_perception.pdf}
    \caption{Respondents risk evaluation of different risk scenarios}
    \label{fig:risk perception}
\end{figure}

\section{Bivariate analysis}


\section{Control group analysis}
