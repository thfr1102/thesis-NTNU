\chapter{Introduction}
\label{chap:intro}

\section{Topic covered by the project}
These days, more and more people are filling their homes with smart devices that ought to make their lives easier. However, many of these devices are connected to the internet, which opens them up to a plethora of risks and attack vectors. Even the devices that are not connected to the internet have risk factors as well. Therefore, the topic of this thesis will look at the security awareness of people who utilises smart home Internet of Things (IoT) devices, as well as identify risks perceptions these users have when living in a smart home. I will explain the background of the topics further in chapter \ref{chap:background}. 

\section{Keywords}
Security awareness, Smart home, IoT, Risk perceptions. 

\section{Problem description}
A smart home is a relatively new concept for many people. Ideally, the users should interact with the devices in a secure manner and understand the risk involved with doing so. However, this security awareness may not have matured in most people yet. This thesis will, therefore, seek to identify the current security awareness level of smart home users, as well as analyse risk perceptions that they might have, in order to enhance current security awareness programs, help vendors prioritise security features, and overall increase security awareness of consumers. These objectives include the identification of usage patterns and user motivations that can impose risk amplification in their daily lives, and also which self-reported security aspects that are considered most important to people. I also want to look at the balance between functionality and risk awareness to identify if the risk is accepted or if the users are oblivious to the risk. 

\section{Justification, motivation and benefits}
With more and more households including smart home devices in their homes, the potential risk increases. In many aspects, the market for smart devices is increasing faster than what the security can keep up with. In 2016 the DNS provider Dyn was targeted by one of the largest DDoS attack in history \cite{wiki:Dyn}. The origin of this attack was an IoT botnet comprising, among other things, smart home devices like IP cameras, printers, and baby monitors. The botnet got access to the devices through brute-forcing default credentials that had not been changed by the user. Incidents like this reveal the severity of the issue. There are also issues that impacts the smart home owners in particular. For example, having more devices to keep up to date, more devices to keep track of credentials on, larger attack surface, and higher consequence if an attacker get access to your home network. Therefore, this thesis will be about uncovering if users are aware of these types of issues and assess how they perceive these risks. This will be done to help both consumers to be more aware of the risks of owning a smart home, as well as provide security professionals some data on what to focus awareness training on. It will also help vendors explore what the consumers think are most important things to focus on securing. 

\section{Research questions}
\label{research:questions}
Based on my problem description, I have identified a couple of research questions I want to explore when performing my project. The research questions are the following: 
\begin{enumerate}
    \item What is the current security awareness level of smart home users in Norway?
    \item What are the most common pitfalls of smart home users in Norway which impose risk amplification?
    \item What do smart home users in Norway perceive being the highest security risks when using smart home devices?
\end{enumerate}

\section{Planned contributions}
In this section I have compiled the contributions that are made in this thesis into tasks that will be performed during this thesis. These tasks are as follows:
\begin{description}
    \item[\textbf{Task 1:}] Identification of definitions to smart homes and security awareness. 
    \item[\textbf{Task 2:}] Identification of related work regarding my research questions mentioned in section \ref{research:questions}. 
    \item[\textbf{Task 3:}] Analysis of the current security awareness level of smart home users in Norway.
    \item[\textbf{Task 4:}] Analysis of current usage patterns, motivations, and routines with negative security impact of smart home users in Norway.
\end{description}

\section{Structure of the thesis}
The report will start with a brief elaboration of the two main topics of my thesis in chapter \ref{chap:background}. This is to provide a foundation to build on, and to make sure we are on the same page regarding certain concepts in relation to the thesis. Further, in chapter \ref{chap:related_work} I will identify and explain the related work surrounding the concepts of my research questions in particular, in order to identify what has already been researched, and what parts of my research questions need further analysis. Moving on to chapter \ref{chap:method}, I will describe the methodology I used to achieve the results, as well as why I chose the methods I used. The results I got from the method are presented in chapter \ref{chap:results}, followed by a discussion on the results in chapter \ref{chap:discussion}, which also include certain limitations to my thesis. Lastly I summarise my thesis in the conclusion in chapter \ref{chap:conclusion} along with possible avenues for future work. 

\begin{comment}
Over the years, several thesis templates for \LaTeX{} have been developed by different groups at NTNU. Typically, there have been local templates for given study programmes, or different templates for the different study levels – bachelor, master, and PhD.\footnote{see, e.g., \url{https://github.com/COPCSE-NTNU/bachelor-thesis-NTNU} and \url{https://github.com/COPCSE-NTNU/master-theses-NTNU}}

Based on this experience, the Community of Practice in Computer Science Education at NTNU (CoPCSE$@$NTNU)\footnote{\url{https://www.ntnu.no/wiki/display/copcse/Community+of+Practice+in+Computer+Science+Education+Home}} is hereby offering a template that should in principle be applicable for theses at all study levels. It is closely based on the standard \LaTeX{} \texttt{report} document class as well as previous thesis templates. Since the central regulations for thesis design have been relaxed – at least for some of the historical university colleges now part of NTNU – the template has been simlified and put closer to the default \LaTeX{} look and feel.

The purpose of the present document is threefold. It should serve (i) as a description of the document class, (ii) as an example of how to use it, and (iii) as a thesis template.
\end{comment}