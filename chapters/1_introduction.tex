\chapter{Introduction}
\label{chap:intro}

\section{Topic covered by the project}
These days, more and more people are filling their homes with smart devices that ought to make their lives easier. However, these devices are connected to the internet, which opens them up to a plethora of attack vectors. This thesis will look at the social aspect of security awareness in people who utilises smart home Internet of Things (IoT) devices, as well as uncover risks perceptions these users have when living in a smart home. 

\section{Keywords}
Security awareness, Smart home, IoT, Risk perceptions. 

\section{Problem description}
A smart home is a relatively new concept for many people. Ideally, the users should interact with the devices in a secure manner and understand the risk involved with doing so. However, this security awareness may not have matured in most people yet. This thesis will, therefore, seek to identify the current security awareness level of smart home users, as well as analyse risk perceptions that they might have, in order to enhance current security awareness programs. These objectives include the identification of usage patterns and user motivations that can impose risk amplification in their daily lives, and also which self-reported security aspects that are considered most important to people. I also want to look at the balance between functionality and risk awareness to identify if the risk is accepted or if the users are oblivious to the risk. 

\section{Justification, motivation and benefits}
With more and more households including smart home devices in their homes, the potential for something bad to happen increases. In fact, it has already happened. In 2016 the DNS provider Dyn was targeted by one of the largest DDoS attack in history \cite{wiki:Dyn}. The origin? An IoT botnet comprising, among other things, smart home devices like IP cameras, printers, and baby monitors. The botnet got access to the devices through brute-forcing default credentials that had not been changed by the user. Incidents like this reveal the severity of the issue, and the reason why this thesis will be about uncovering if users are aware of these types of issues and whether they perceive these risks or not. 

\section{Research questions}\label{research:questions}
\begin{itemize}
    \item What is the current security awareness level of smart home users in Norway?
    \item What are the most common pitfalls of smart home users in Norway which impose risk amplification?
    \item What do smart home users in Norway perceive being the highest security risks when using smart home devices?
\end{itemize}

\section{Planned contributions}
The following is a description of the tasks and contributions I will perform and make in my thesis: 
\begin{itemize}
    \item[\textbf{Task 1:}] Identification of definitions to smart homes and security awareness. 
    \item[\textbf{Task 2:}] Analysis of the current state of the art, in respect to assessment methodologies of security awareness.
    \item[\textbf{Task 3:}] Analysis of the current security awareness level of smart home users.
    \item[\textbf{Task 4:}] Analysis of current usage patterns, motivations, and routines with negative security impact of smart home users.
\end{itemize}

\begin{comment}
Over the years, several thesis templates for \LaTeX{} have been developed by different groups at NTNU. Typically, there have been local templates for given study programmes, or different templates for the different study levels – bachelor, master, and PhD.\footnote{see, e.g., \url{https://github.com/COPCSE-NTNU/bachelor-thesis-NTNU} and \url{https://github.com/COPCSE-NTNU/master-theses-NTNU}}

Based on this experience, the Community of Practice in Computer Science Education at NTNU (CoPCSE$@$NTNU)\footnote{\url{https://www.ntnu.no/wiki/display/copcse/Community+of+Practice+in+Computer+Science+Education+Home}} is hereby offering a template that should in principle be applicable for theses at all study levels. It is closely based on the standard \LaTeX{} \texttt{report} document class as well as previous thesis templates. Since the central regulations for thesis design have been relaxed – at least for some of the historical university colleges now part of NTNU – the template has been simlified and put closer to the default \LaTeX{} look and feel.

The purpose of the present document is threefold. It should serve (i) as a description of the document class, (ii) as an example of how to use it, and (iii) as a thesis template.
\end{comment}