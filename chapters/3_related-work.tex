\chapter{Related work}
This chapter will go over related work that has been previously done on the topic of my thesis. I will systematically go through my research questions and uncover topics that are relevant to them when talking about the literature that is out there. I will delve into each of my research questions to figure out to what extent information in the literature can provide answers to the research questions I identified, and which areas or research questions the literature provides insufficient information.

\section{RQ1: What is the current security awareness level of smart home users in Norway?}
The paper by Kang et al. \cite{Kang2015}, makes use of mental models to assess the participants knowledge of the Internet, and how the level of knowledge affect their privacy and security decisions. People with more articulated mental models perceived more privacy threats, possibly because of better knowledge on where the specific threats could occur. However, the study did not find any connection between peoples technical background and the security measures taken to control their security and privacy online. Mental models could be an interesting method to consider when assessing the knowledge of a smart home, and the relation this has to a participant's security awareness levels. 

The aim of the paper by Drevin et al. \cite{Drevin2006} is to introduce a value-focused assessment methodology when identifying ICT security awareness aspects. The approach focuses on identifying the stakeholders that would be impacted by the decisions, and questioning them about their values related to the area of interest. These values are then used to identify objectives like maximising the confidentiality and integrity of data. For my thesis it could be interesting to consider taking a value-based approach when assessing the security awareness and especially the risk perceptions of smart home users. 

In a study by McReynolds et al. \cite{McReynolds:2017:TLS:3025453.3025735} they focus specifically on the privacy concerns, expectations and security awareness of using connected toys and gadgets for the home that can listen to you speak. As they are waiting for voice commands, they are similar to voice assistants in that they blend into the background and are always listening. The study consisted of interviews with parent-child pairs in which they interacted with common connected toys. The results were that the children were often unaware that others might be able to hear what was said around the toy, and many parents voiced privacy concerns. 

Another paper by Gerber et al. \cite{Gerber2018} also focuses on the privacy threats of a smart home. They found that most people were unable to state even a single privacy consequence, and most people listed quite general privacy issues like profiling and data collection, but also threats not related to privacy in particular. 

There exists prior studies that also focus on security awareness in Norway specifically. Gunleifsen \cite{GunleifsenHakon2018Csaa} addresses the level of security awareness, perception, culture of users of ICT in Norway, and whether it can be improved. The paper covers different aspects of security awareness, such as general security knowledge, self-evaluation of risk, and different behavioural patterns in regards to WiFi connections, authentication routines and phishing awareness. The findings can be summarised with the fact that the level of security awareness can be significantly improved, however the results were better than similar national studies. 

A study by Ghiglieri et al. \cite{Ghiglieri2017} focused on exploring consumer awareness and attitudes of Smart TV related privacy risks. The study was conducted in three steps with questionnaires. The first aimed to assess the awareness of privacy related risks of using a Smart TV, which showed a meagre level of awareness. The main findings of the second part include that the consumers were generally unwilling to give up the functionality of a Smart TV for the sake of privacy. Lastly, respondents were asked to choose between five different Smart
TV Internet connection options, in which two retained functionality, however included using extra time and effort to preserve privacy. The results from this showed that they were willing to use some extra time and effort, but only if the functionality was not impaired. 

Another paper also focuses on the consumer perspective, regarding awareness of botnet activity of consumer IoT devices. In the study by McDermott et al. \cite{McDermott2019} they assessed user ability to detect threats in their smart devices. The conclusion was that it was very difficult for the consumers to detect and be aware of whether or not a device was infected without any clear signs. Interestingly, they also discovered that there were no correlation between level of technical knowledge and ability to detect these infections. 



\section{RQ2: What are the most common pitfalls of smart home users in Norway which impose risk amplification?}
Not many papers focus specifically on the mistakes or bad habits of smart home users, however, these bad habits could somewhat be inferred by combining security awareness research and vulnerability assessments on smart homes.

In a recent article, Awad and Ali \cite{Awad2018} seeks to identify possible security risks in order to understand the current security status of smart homes. They apply the operationally critical threat, asset, and vulnerability evaluation (OCTAVE) methodology, which focus on information assets in relation to different information containers. The main results from this assessment were a list of threats ordered by risk score. The highest scoring threats were related to unauthorised access and execution of operations, as well as loss of control. 

Another article, by Denning et al. \cite{Denning:2013:CSM:2398356.2398377}, also highlights security risks associated with using home technologies. The article explores the landscape of technological attacks on smart homes, identified key features in devices that make them more vulnerable and human assets at stake. Using these three concepts they applied their framework to three example technologies, a wireless webcam toy, a wireless scale, and a home automation siren. This framework can be used to determine risk areas of different devices, and therefore also potential pitfalls the users can fall into. 

A paper by Caviglione et al. \cite{Caviglione2015} analyses the human-related aspects of security and privacy threats in smart environments and reviews the major risks arising from using such devices, emphasising networking. It takes a role-based approach, focusing on vendors, customers, operators and deployers. The results show that each group have their own pit-falls, and for customers this is projected as lack of awareness. This, in turn, affect the other roles as customers will not demand better security on their products, since they are unaware of the insufficiency. It also emphasises that security should come from the other groups than customers, since awareness campaigns have had little effect. 

\section{RQ3: What do smart home users in Norway perceive being the highest security risks when using smart home devices?}
First lets look at some studies about risk perception in general. A quantitative empirical study by Schaik et al. \cite{Schaik2017} analysed the perceptions of risk a number of students had towards a set of 16 different security risks. The results of the study concluded that the highest perceived risks were identity theft, keyloggers, cyber-bullying, and social engineering. It also identified predictors of perceived risk, which were voluntariness, immediacy, catastrophic potential, dread, severity of consequences and control, as well as Internet experience and frequency of Internet use. Control was also a significant predictor of precautionary behaviour. 

Another paper, written by Conti and Sobiesk \cite{Conti2007}, aims to identify user perceptions on web-based information disclosure. The paper assumes that we face a growing tension between privacy concerns of individuals and financial motivations of organisations, and seeks to explore these issues through querying students about their risk perceptions. The results can be summarised that the students believe that an honest man has nothing to fear, which were mostly contradictory to beliefs of security and privacy professionals. This result is similar to the issue raised by Solove \cite{Solove2007}, where he tries to break the argument apart and counter it. 

Some studies have also been conducted with focus on risk perception in smart homes, like the article by Zeng et al. \cite{Zeng:2017:EUS:3235924.3235931}. It focuses on the disjointed perception of risk between the end users and security experts, and was conducted using semi-structured interviews with 15 people living in smart homes. Similarly to many other studies mentioned in this chapter, it utilised mental and threat models to assess security awareness. The results included a gap in threat models due to limited technical understanding and awareness of some security issues but limited concern. The study also revealed that the participants have varied threat models and do no share a common set of concerns when it comes to risk perceptions. However some of the threats were video/audio recording, adversarial remote control, network attack, spying by other users in home, and account/password hacking. 

As the previous paper slightly touches on, there also seem to be a concern that people in the same household violates each others privacy. The paper by Ur et al. \cite{Ur:2014:IVI:2632048.2632107} focuses specifically on how deployment of connected locks and security cameras in a smart home may impact a teenager's privacy, and in turn the relationship between parent and teen. They conducted a series of interviews with teenagers and parents and investigated reactions to audit logs of family members. The parents wanted audit logs with photos, but teenagers preferred only text logs or no logs at all, and were averse to include photos. 

Another paper \cite{Tangstad2017} written by students at the University of Tromsø touches on risks related to the procurement of a smart home, specifically about perceived risk from privacy and security issues. They asked questions about how much users trust that the data security and privacy is safeguarded in a smart home system, and how much this affect their willingness to procure a smart home. The conclusion is that most respondents are either sceptical or unsure as to whether the smart home safeguards their privacy and security. It also shows that most people do take into consideration the privacy and security of the smart home system before procuring it for themselves. 

A study by Brush et al. \cite{Brush2011} sought to get insight into the challenges and opportunities of home automation, in order for smart homes to become amenable for broader adoption. They conducted a series of home visits to households with home automation, and identified four barriers. These were high cost of ownership, inflexibility, poor manageability, and difficulty of achieving security. For the security barrier, the participants were especially worried that remote access to their smart devices introduced security risk, even though the functionality was very appealing. 

There has also been some studies regarding risk perceptions of IoT security in other aspects of society, especially regarding critical social services. In particular, a study by Asplund and Nadjm-Tehrani \cite{Asplund2016} presents the perceptions and attitudes on the security of IoT and relates them to the current challenges of IoT in general. The paper demonstrated optimism in the utility of such devices, however there was a lack of consensus regarding the risks. It also showed that many people did not believe there are any significant risks associated with IoT, since the risk factors are already accounted for in regular system design. 

In another article by Gerber et al. \cite{Gerber2019} peoples privacy risk perceptions were assessed in relation to, but not limited to, smart homes. They found that when users assess their risk perception, they are more likely to perceive higher risk from more specific scenarios, whereas abstract scenarios were deemed less severe. This could mean that people does not seem aware of specific privacy risks when confronted with an abstract risk scenario. 