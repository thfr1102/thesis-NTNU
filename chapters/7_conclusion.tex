\chapter{Conclusion}
\label{chap:conclusion}


\section{Future work}


\begin{comment}
Råd:
1. Repeter problemstillingen.
2. Formuler de generelle og overordnede funnene (mønstrene) i data materialet. 
3. Tenk deg disposisjonen i konklusjonen som en trakt satt på hodet - du begynner med det spesielle (problemstillingen) og avslutter med det generelle (det store perspektivet). 
4. La konklusjonen bli svar på spørsmålet "Hva har vi nå fått (ny) kunnskap om?"
5. Si alt klart og tydelig, ikke overlat til leserne å lese mellom linjene, selv om du synes du har sagt det før.
6. Viktig informasjon må gjentas. 
7. Bring aldri ny informasjon inn i konklusjonen. Alt som står der, må være omtalt tidligere. 
8. Gå aldri inn på detaljer i en konklusjon.
9. Vær forsiktig med bastante sannheter. Velg ord som gir inntrykk av en viss forsiktighet, for eksempel antyde, tyde på, observere, finne, resultere i, kan være, til en viss grad, kanskje, i mange tilfeller, ofte osv.
\end{comment}