\chapter{Discussion}
\label{chap:discussion}
This chapter will focus on discussing the different results from the previous chapter. The chapter will focus on separating immaterialities, details, doubts, and other things to be able to reach a conclusion. The discussion is structured around the sample and the three research questions of my thesis, and the discussion is conducted with those in mind. Lastly, the limitations of the study as a whole will be discussed. 

\section{Sample representativeness for smart home users in Norway}
In the previous chapter, I went through the demographics data and saw that only peoples residency was similar to the overall population, based on SSB data \cite{SSB_befolkning}. For age, there was a massive overweight of people between 30 and 49 years old, and way fewer older people above 60 and younger people below 20. In addition to this, my sample only contained men, and had an abnormally high number of people who has done vocational college. Whether this sample is representative of the overall smart home users in Norway is hard to say since I only collected data from a single Facebook group. I of course also collected data from a control group, which showed a different demographic makeup, especially in gender and age, but also education, eluding to the fact that the main sample may not be representative to every smart home user. On the other hand, the sample may be representative for those especially interested in the smart home ecosystem, since my results in section \ref{howsmarthome} show that the main sample have, on average, much smarter homes. The control group sample mostly said they had generic smart home devices like a SmartTV, and not fully integrated smart home systems. The main sample also reported overall higher technical knowledge than the control group.

\section{RQ1: What is the current security awareness level of smart home users in Norway?}
For my first research question, the aim was to assess the security awareness of smart home users in Norway. To answer this I asked a series of questions in the form of an online survey to people in a smart home enthusiast Facebook group for Norwegian smart home users, although anyone could join and therefore take part in the survey. In fact, from my results, one person claimed to not owning any smart home devices. When trying to assess the security awareness of the smart home users, I focused on their use of the smart devices, their credential management routines, knowledge of certain security aspects, and their risk perceptions. The questions that aims to answer these aspects was mostly based on the best practises for security awareness by ENISA \cite{ENISA2015SmartHome}. Initially, I came in with the hypothesis that their routines regarding the use of smart home devices carried significant risk, and that their security awareness is generally low. However, as I argue in the coming sections, this turned out to not be entirely true. 

For example, when asking about their routines when updating their devices, I observed that 62.2\% update them right away, and another 34.2\% do so, but sometimes wait a while. Only 3.6\% did not think about it that much, which is a very small amount of people. This shows us that the overwhelming majority of people actually care about updating their devices, even though approximately one third still wait a while before updating. The results are similar for the control group, however there are slightly fewer people updating right away (46.5\%) and slightly more people (41.9\%) sometimes waiting a while. 

Another example is when I asked if they turn off services and features they do not use, which resulted in about two thirds (66.2\%) confirming that they did so. This also denied my initial hypothesis that this was something the respondents mostly did not do. My results are also consistent with the control group sample, which showed that 60.5\% do and 32.6\% do not, while the rest did not know. 

In addition to the questions mentioned previously, I also asked if they change their privacy and security settings on their smart devices. And as a matter of fact, most people (56.3\%) do, while 38.7\% do not. This is also mostly consistent with the control group, where 45.7\% do, and 48.6\% do not, which is a difference in about 10\%. 

Another aspect was whether the respondents preferred to use cable or wireless when connecting their smart home devices to the internet. According to ENISA \cite{ENISA2015SmartHome}, the best practise is to use cable where it is possible due to the smaller attack surface. My initial hypothesis was that the sample would prefer wireless, since this is easier to set up and use, while also being more visually pleasing by not having cables lying around. This turned out to be false however, as 64\% preferred cable, only 20.7\% preferring wireless, and the rest answering that it was not important to them. This was also one of the questions I was curious about regarding my control group, since the two samples have a different technical background, and the control group on average having less smart home devices. The results from the control group showed a complete opposite distribution, were 65.7\% preferred wireless and 28.6\% preferred cable. It could be that these differences are due to the discrepancies in technical knowledge and how smart their homes are, since according to my results, most people in the control group only had very basic smart home devices. Another finding was that there were differences in the age groups when it comes to internet connection preference. People older than 50 almost equally preferred cable and wireless, while the other age groups heavily preferred cable, which may indicate that older people do not think about the security aspects while choosing. Another possible explanation is that the reason why older people choose to implement smart home devices in their homes is mainly to make their lives easier, and wireless connection is a much easier and more intuitive option for connecting to the internet. 

On the other hand, I also asked if the respondents used a separate segment of their home for their smart home devices. A separate segment for smart home devices is considered best practise so that these devices do not have direct contact with the personal devices of the household, like smart phones, laptops, tablets and so on. My results show that most people do not connect these devices to a separate segment, where 59\% answered no, and 40.5\% answered yes. The difference here is not very large, however, there is definitely room for improvement. This difference is especially jarring when it comes to the control group, which only had slightly more people saying no, but this difference was minuscule. However, the big difference is shown when lot of people (20\%) in the control group said they did not know, which may indicate that they do not know how to segment their network in the first place. 

Another indication of their security awareness level is how they manage their credentials. When asked about their routines of changing the default password on recently purchased devices, they overwhelmingly said that they do, with 83.8\% saying yes. On the other hand, when asked about using password managers, about half of the respondents said yes and no. For the control group, there were less people saying yes, and more people saying that they do not know about password managers, following the narrative that the control group are slightly behind the main sample in overall terms of security awareness. Another classic indication of bad credential management is whether they use the same password on multiple services and devices. In my results, only a few people (7.7\%) admitted to using the same password everywhere, while 36.9\% reused their password on a couple of services. On the one hand, this is certainly not ideal, but seems decent when compared to the control group, in which 65.1\% used the same password on a couple of services and devices as is displayed in figure \ref{controlgroup_samepassword}.

\section{RQ2: What are some of the most common pitfalls of smart home users in Norway which impose risk amplification?}
%\cite{ESET2016}

\section{RQ3: What do smart home users in Norway perceive being the highest security risks when using smart home devices?}


\section{Strengths and limitations}


\begin{comment}
Råd:
1. Begynn drøftingen med å repetere problemstillingen. Referer deretter de viktigste funnene fra teorikapitlet.
2. Skill tydelig mellom andres syn og ditt eget. Bruk helst vi-form når du fremmer egne synspunkter, referatmarkører når du refererer kildene.
3. Marker at du diskuterer ved å bruke uttrykk som "På den ene siden ... på den andre siden", "mot dette kan det hevdes at", "noen hevder ... andre sier tvert imot at".
4. Sørg for å gjøre klart hvilke kriterier du vurderer ut fra.
5. La drøftingen ta utgangspunkt i en hypotese, det vil si en påstand du vil undersøke nærmere. 
6. Tenk deg drøftingen som en pro-et-contra-oversikt med argumenter for (pro) på den ene siden av arket og argumenter imot (contra) på den andre siden. 
7. Husk å lage en overgang til konklusjonen ved å spørre deg selv følgende spørsmål: "Hvilke(n) konklusjon(er) kan trekkes av denne drøftingen".
\end{comment}